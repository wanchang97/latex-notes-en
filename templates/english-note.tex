% templates/english-note.tex
\documentclass[11pt]{article}
\usepackage{../styles/notes-en}

\title{Note Title}
\author{Your Name}
\date{\today}

\begin{document}
	
	\maketitle
	
	\begin{abstract}
		This is a brief summary of the note content.
	\end{abstract}
	
	\tableofcontents
	
	\section{Introduction}
	
	Start writing your note here. This template provides all the essential features for technical note-taking.
	
	\section{Mathematics}
	
	\subsection{Inline Equations}
	Einstein's mass-energy equivalence: $E = mc^2$.
	
	\subsection{Display Equations}
	Quadratic formula:
	\[
	x = \frac{-b \pm \sqrt{b^2 - 4ac}}{2a}
	\]
	
	Integral definition:
	\[
	\int_a^b f(x)dx = F(b) - F(a)
	\]
	
	\section{Mathematical Environments}
	
	\begin{definition}[Derivative]
		The derivative of a function $f(x)$ at point $x_0$ is defined as:
		\[
		f'(x_0) = \lim_{h \to 0} \frac{f(x_0 + h) - f(x_0)}{h}
		\]
	\end{definition}
	
	\begin{theorem}[Pythagorean Theorem]
		In a right triangle, the square of the hypotenuse equals the sum of the squares of the other two sides:
		\[
		a^2 + b^2 = c^2
		\]
	\end{theorem}
	
	\begin{proof}
		This can be proven using geometric area methods.
	\end{proof}
	
	\begin{example}
		Calculate the derivative of $f(x) = x^2$ at $x=2$.
	\end{example}
	
	\begin{remark}
		This is an important remark about the content.
	\end{remark}
	
	\section{Code Examples}
	
	\subsection{Python Code}
	\begin{lstlisting}[style=python, caption=Python Function Example]
		import numpy as np
		
		def calculate_quadratic(a, b, c, x):
		"""Calculate quadratic function value"""
		return a*x**2 + b*x + c
		
		# Generate sample data
		x_values = np.linspace(-10, 10, 100)
		y_values = calculate_quadratic(1, -3, 2, x_values)
		
		print("Calculation completed successfully")
	\end{lstlisting}
	
	\subsection{JavaScript Code}
	\begin{lstlisting}[style=javascript, caption=JavaScript Function Example]
		function fibonacci(n) {
			if (n <= 1) return n;
			return fibonacci(n - 1) + fibonacci(n - 2);
		}
		
		// Calculate Fibonacci number
		console.log(`Fibonacci(10) = ${fibonacci(10)}`);
	\end{lstlisting}
	
	\section{Note Environments}
	
	\begin{note}
		This is a regular note box for additional information or explanations.
	\end{note}
	
	\begin{important}
		This is an important note box for highlighting key concepts or critical information.
	\end{important}
	
	\begin{warning}
		This is a warning box for cautionary notes or potential pitfalls.
	\end{warning}
	
	\section{Lists and Organization}
	
	\subsection{Ordered List}
	Steps to learn LaTeX:
	\begin{enumerate}
		\item Install LaTeX distribution
		\item Learn basic syntax
		\item Write your first document
		\item Use advanced features like mathematics and code highlighting
	\end{enumerate}
	
	\subsection{Unordered List}
	Benefits of using LaTeX for notes:
	\begin{itemize}
		\item Professional typesetting
		\item Excellent mathematical formula support
		\item Version control friendly
		\item Cross-platform compatibility
	\end{itemize}
	
	\section{Tables}
	
	\begin{table}[htbp]
		\centering
		\begin{tabular}{lccc}
			\toprule
			Algorithm & Time Complexity & Space Complexity & Accuracy \\
			\midrule
			Linear Regression & O(n) & O(1) & 85\% \\
			Random Forest & O(n log n) & O(n) & 92\% \\
			Neural Network & O(n^2) & O(n^2) & 95\% \\
			\bottomrule
		\end{tabular}
		\caption{Comparison of machine learning algorithms}
		\label{tab:algorithm_comparison}
	\end{table}
	
	\section{Conclusion}
	
	This note demonstrates the following features:
	\begin{itemize}
		\item Mathematical formulas and environments
		\item Code highlighting for multiple languages
		\item Custom note environments
		\item Professional table formatting
		\item Automated deployment to GitHub Pages
	\end{itemize}
	
\end{document}